\chapter{Resolving links}\label{resolving-links}

\section{Server configuration}\label{server-configuration}

The server configuration file installed in \S\ref{configuration-files} defines the first steps the server has to take to respond to the users' petitions.
Every time a user tries to access the service, they do it with an HTTPS petition to \url{https://sl.ugr.es/}.
When the server receives a petition, it opens the \texttt{location /} block in the configuration file and looks inside it to find a page they can serve the user to respond to their petition:

\begin{lstlisting}
location / {
	try_files $uri /php/resolver.php?$args;
	index pages/index.php;
}
\end{lstlisting}

The \texttt{try\_files} directive tries to find a find a page corresponding to the requested URI.
For example, if the user asks for \texttt{https://sl.ugr.es/licence}, it will try to find a page that can be served for the URI \texttt{licence}.
It is guaranteed that \texttt{\$uri} will fail, since all pages are inside the \texttt{pages/} directory, which means that every time the user asks for a page they will be redirected to \texttt{/php/resolver.php}.
The \texttt{?\$args} suffix means that any arguments passed along with the URI will be passed to the resolver so that they can be used by the pages it redirects them to.
We will discuss the resolver with greater depth in \S\ref{the-resolver-script}.
The \texttt{index} directive lets the server know what to load when the user accesses the service directly, i.e. \url{sl.ugr.es}.

The \texttt{location \mysim\char`\\.php\$} block does the same as the previous block, but tries \texttt{\$fastcgi\_script\_name} instead.
This directive is used instead of \texttt{\$uri} because \texttt{fastcgi} is the protocol that effectively loads the PHP service.

Finally, the \texttt{error\_page} directive redirects all \texttt{404} errors to \texttt{/pages/404.php}.
In theory, this directive is not required because the resolver sends lost users to the 404 page, but nobody is going to complain about a little resilience and loading the custom 404 page is much more elegant than serving the default one.

\begin{lstlisting}
error_page 404 /pages/404.php;
\end{lstlisting}

\section{The resolver script}\label{the-resolver-script}

Every petition is forwarded to this script, which parses the URI and finds the best page to serve the user or redirects them from a short link.
The script was written for scalability reasons given the requisites of Neosluger:

\begin{itemize}
\item
	The user must be able to access pages inside the site without having to know the absolute path.
	E.g. \texttt{sl.ugr.es/man} must be able to serve \texttt{pages/manual.php}.
\item
	The user must be able to use a short URL to be automatically redirected and prepend \texttt{stats/} to the URL handle to inspect it and see its usage statistics.
\item
	The users must not be able to access the different pages from their location in the server directory tree.
	This improves the service's security.
\item
	The old API must still work, even though it will always throw an error.
\end{itemize}

If the URIs were resolved by the server configuration file, every time a new page was added to the service it would need to be modified and it would get clobbered really fast.
Instead, we let the resolver script handle this for us and check whether the user is accessing a page within the service or wants to be redirected from a short URL.

The resolver follows these steps to serve the user the page they expect:

\begin{lstlisting}
Parse URI received from the server

if a maching page is found in pages/*.php:
	render the page
otherwise, if the URI is NOT a handle found in the database:
	if it refers to the old API:
		serve an API deprecation error message
	otherwise
		serve the 404 page
otherwise, if the handle was found in the database:
	Prevent the user from caching this page to register all accesses to the link
	Register the access
	Redirect the user to their destination
\end{lstlisting}

The order of the paths is unusual because checking if a URL wasn't found in the database is done by a call to \texttt{\$url->is\_null()} and \texttt{if} statements are written so that their conditions never contain a negation, which is easier to reason about.

\section{Rendering pages}\label{rendering-pages}

All pages in \texttt{pages/*.php} \textbf{must} contain a \texttt{render} function to act as the main function of the page and call them at the end of the file.
This way, when the resolver includes them, the \texttt{render} function is called automatically by the page.

Pages are rendered using Twig, a templating engine that lets Neosluger reuse interface code and be much easier to maintain.
Twig directives are specified inside \texttt{\{\% \%\}} and \texttt{\{\{ \}\}} blocks.
The first type of blocks are used for reserved words like \texttt{[end]block}, \texttt{[end]if} and to call objects' member functions like \texttt{url.is\_null()} (notice the \texttt{.} operator instead of \texttt{->}).
The second type of blocks expands the value inside the braces.
For example, if we have a variable \texttt{\$name = "Taxo"}, \texttt{<p>My name is \{\{ name \}\}.</p>} will be expanded to \texttt{<p>My name is Taxo.</p>}

To render a page, \texttt{Twig->render()} must be called from an \texttt{echo} statement.
It receives the route of the HTML file it has to render and an associative array of objects where the names are the variables used by the template.
