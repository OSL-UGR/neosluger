\chapter{Setting up the server}\label{setting-up-the-server}

\section{Configuration files}\label{configuration-files}

Neosluger is configured to run in an Nginx server.
It can be configured to run with Apache, but this is not officially supported.
The server configuration file is available in Neosluger's repository \texttt{conf/} directory and is named \texttt{neosluger.conf}.
To install it, add it to \texttt{/etc/nginx/sites-enabled/neosluger.conf} and modify it to match you system's settings as required.
The default directory to host Neosluger is \texttt{/var/www/neosluger/src}.
If you decide to place it in any other directory, you must change the \texttt{root} directory to match the new path.
Finally, add the following directive in \texttt{/etc/nginx/nginx.conf}:

\begin{lstlisting}[language=sh]
http {
	# Add this include directive in the body of the already existing http block:
	include /etc/nginx/sites-enabled/neosluger.conf;
}
\end{lstlisting}

% TODO: Ubuntu shenanigans

\section{Dependencies}\label{dependencies}

Neosluger requires \texttt{php}, \texttt{php-gd}, \texttt{php-fpm} \texttt{mongodb} and \texttt{php-mongodb} to run.
You can easily install it with your distro's package manager, but bear in mind that the packages may have different names depending on the repository you are accessing.
After installing PHP, install \texttt{composer} to manage Neosluger's internal dependencies.
The latest version at the time of writing is \texttt{v2.2.6} and it is the version that has been used to develop Neosluger.
If the version provided by your package manager is lower than \texttt{v2.0} or throws warnings when installing the internal dependencies, reinstall \texttt{composer} manually following the official instructions\footnote{
	\url{https://getcomposer.org/download/}
}.

To install Neosluger's internal dependencies, move to the \texttt{src/} directory and install them with \texttt{composer}:

\begin{lstlisting}[language=sh]
cd src/
composer install
\end{lstlisting}

This will install the latest versions of the dependencies.
To install them with the latests tested versions, copy the \texttt{composer.lock} file inside \texttt{src/}, next to where \texttt{composer.json} is.
This file contains metadata about the latest update so that running \texttt{composer install} will download the correct dependencies' versions to match the locked system.
In short, if it works in my machine, I can send you my lock file to make it work in yours too.

\section{Setting up the database}\label{setting-up-the-database}

Neosluger stores all its data in a local MongoDB database.
Its address is defined in \texttt{php/const.php} and points to the local database set up by the MongoDB service.
You can change it if your instance uses a database in a different location.
When the database is up, run \texttt{conf/migrate.php} to set up the database indices.

\subsection{Benchmarking}\label{benchmarking}

You can run \texttt{conf/benchmark.sh} to test the time spent in subsequent API calls.
Run \texttt{conf/benchmark.sh --help} for more details.

\section{Starting the service}\label{starting-the-service}

With the server correctly configured and all dependencies installed, you can now start the \texttt{systemd} services required to run Neosluger:

\begin{lstlisting}[language=sh]
sudo systemctl start nginx php-fpm mongodb
\end{lstlisting}

You can also use the following command to automatically report errors:

\begin{lstlisting}[language=sh]
sudo systemctl restart nginx php-fpm mongodb || sudo systemctl status nginx
\end{lstlisting}

After this, you should be able to access the server with the IP address configured by Nginx.

\section{Cache directory}\label{cache-directory}

QR codes are saved in a cache directory to be presented to the user and be downloaded by them.
Its route is set in \texttt{pages/url-result.php} and defaults to \texttt{src/cache/qr/}.
You may be forbidden from writing into this directory if Neosluger is hosted in \texttt{/usr/}.
If that is the case, please refer to this ServerFault answer: \url{https://serverfault.com/a/997496}.

To use the directory, you need to give the user running PHP permission to write into it.
Neosluger will create the cache directory if it does not exist, so you should give write permissions to \texttt{src/}.
This may look like the following:

\begin{lstlisting}[language=sh]
chown -R "$USER":http src/
chmod -R g+w src/
\end{lstlisting}

\section{Testing the service}\label{testing-the-service}

Once Neosluger is running in your server, you can run its test suites using PHPUnit to ensure that the system won't throw any errors at runtime:

\begin{lstlisting}[language=sh]
cd src/
vendor/bin/phpunit tests
\end{lstlisting}

The test suites also acts as a complete reference for any developer who wishes to extend the system.

\subsection{Directory permissions clash}

The web server is ran by the user \texttt{http}, while the tests are ran by \texttt{"\$USER"}.
This means that whoever gets to create the cache directory first gets all the permissions to read and write to it.
To solve this, you \textbf{must} edit the cache directory's owners in two cases:

\begin{itemize}
\item
	After creating the cache directory from running the test suites so that the web server can create the codes.
\item
	After creating the cache directory from a web server response so that the tests can pass.
\end{itemize}

To update the owners you just need to \texttt{chown} the deepest directory:

\begin{lstlisting}[language=sh]
cd src/
chown "$USER":http cache/qr
\end{lstlisting}

The directory's permissions are set to \texttt{775} by the service.
