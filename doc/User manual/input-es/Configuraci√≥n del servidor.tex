\chapter{Configuración del servidor}\label{configuracion-del-servidor}

\section{Ficheros de configuración}\label{ficheros-de-configuracion}

Neosluger está configurado para ejecutarse en un servidor Nginx.
Se puede configurar para usarlo con Apache, pero esta opción no tiene soporte oficial.
Puedes encontrar la configuración del servidor en el repositorio bajo el nombre de \texttt{neosluger.conf} dentro del directorio \texttt{conf/}.
Para instalarlo, debes copiarlo en \texttt{/etc/nginx/sites-enabled/neosluger.conf} modificar lo que necesites para que concuerde con la configuración de tu sistema.
El directorio por defecto para alojar Neosluger es \texttt{/var/www/neosluger/src}.
Si decides colocarlo en cualquier otro directorio, debes modificar el directorio \texttt{root} para que concuerde con la ruta nueva.
Por último, añade la siguiente directiva a \texttt{/etc/nginx/nginx.conf}:

\begin{lstlisting}[language=sh]
http {
	# Añade esta directiva include en el cuerpo del bloque http ya existente:
	include /etc/nginx/sites-enabled/neosluger.conf;
}
\end{lstlisting}

% TODO: Ubuntu shenanigans

\section{Dependencias}\label{dependencias}

Neosluger requiere los paquetes \texttt{php}, \texttt{php-gd}, \texttt{php-fpm} \texttt{mongodb} and \texttt{php-mongodb} para funcionar.
Puedes instalarlos fácilmente con el gestor de paquetes de tu distro, pero ten en cuenta que los paquetes pueden diferir en el nombre dependiendo del repositorio al que estés accediendo.
Tras instalar PHP, instala \texttt{composer} para gestionar las dependencias internas de Neosluger.
La última versión a fecha de redacción es \texttt{v2.2.6} y es la versión que se ha utilizado para desarrollar Neosluger.
Si la versión provista por tu gestor de paquetes es inferior a \texttt{v2.0} o lanzar \textit{warnings} durante la instalación, reinstala \texttt{composer} manualmente siguiendo las instrucciones oficiales\footnote{
	\url{https://getcomposer.org/download/}
}.

Para instalar las dependencias internas de Neosluger, muévete al directorio \texttt{src/} e instálalas con \texttt{composer}:

\begin{lstlisting}[language=sh]
cd src/
composer install
\end{lstlisting}

Esto instalará la última versión de las dependencias.
Para instalarlas con las últimas versiones probadas copia el fichero \texttt{composer.lock} dentro de \texttt{src/} junto a \texttt{composer.json}.
Este fichero contiene metadatos sobre la última actualización, de forma que al ejecutar \texttt{composer install} se descargarán las versiones correctas de las dependencias para que coincidan con el sistema bloqueado.
En resumen, si funciona en mi máquina, puedo enviarte mi fichero \texttt{lock} para que también funcione en la tuya.

\section{Configuración de la base de datos}\label{configuración-de-la-base-de-datos}

Neosluger guarda todos sus datos en una base de datos MongoDB.
Su dirección está definida en \texttt{php/const.php} y apunta a la base de datos local iniciada por el servicio de MongoDB.
Puedes modificarla si tu instancia utiliza una base de datos en una ubicación diferente.
Cuando la base de datos esté funcionando, ejecuta \texttt{conf/migrate.php} para configurar los índices.

\subsection{\textit{Benchmarks}}\label{benchmarks}

Puedes ejecutar \texttt{conf/benchmark.sh} para probar el tiempo utilizado en llamadas subsguientes a la API.
Ejecuta \texttt{conf/benchmark.sh --help} para más detalles.

\section{Inicio del servicio}\label{inicio-del-servicio}

Con el servidor correctamente configurado y todas las dependencias instaladas puedes iniciar los servicios de \texttt{systemd} requeridos para ejecutar Neosluger:

\begin{lstlisting}[language=sh]
sudo systemctl start nginx php-fpm mongodb
\end{lstlisting}

También puedes usar la siguiente orden para reportar errores automáticamente:

\begin{lstlisting}[language=sh]
sudo systemctl restart nginx php-fpm mongodb || sudo systemctl status nginx
\end{lstlisting}

Tras esto, deberías poder acceder al servidor con la dirección IP configurada por Nginx.

\section{Directorio de caché}\label{directorio-de-cache}

Los códigos QR se almacenan en un directorio de caché para presentarlos al usuario y que éste los descargue.
Su ruta está definida en \texttt{pages/url-result.php} y apunta por defecto a \texttt{src/cache/qr/}.
Puede que tengas prohibido escribir en este directorio si Neosluger está alojado en \texttt{/usr/}.
Si éste es el caso, échale un vistazo a esta respuesta de ServerFault \url{https://serverfault.com/a/997496}.

Para usar el directorio necesitas darle permisos de escritura al usuario que ejecuta PHP.
Neosluger creará el directorio si no existe, así que debes darle los permisos en \texttt{src/}.
Puedes hacerlo de la siguiente forma:

\begin{lstlisting}[language=sh]
chown -R "$USER":http src/
chmod -R g+w src/
\end{lstlisting}

\section{Pruebas sobre el servicio}\label{pruebas-sobre-el-servicio}

Cuando Neosluger esté ejecutándose en tu servidor, puedes ejecutar las pruebas usando PHPUnit para asegurarte de que el sistema no te lanzará errores en tiempo de ejecución:

\begin{lstlisting}[language=sh]
cd src/
vendor/bin/phpunit tests
\end{lstlisting}

Las pruebas también sirven como una referencia completa para cualquier desarrollador que desee extender el sistema.

\subsection{Choque de permisos sobre directorios}

El servidor web se ejecuta por el usuario \texttt{http} mientras que los tests los ejecuta \texttt{"\$USER"}.
Esto implica que quien cree antes el directorio de caché se queda con todos los permisos de lectura y escritura.
Para solucionarlo, \textbf{debes} editar los dueños del directorio de caché en dos casos:

\begin{itemize}
\item
	Tras crear el directorio de caché al ejecutar las pruebas para que el servidor web pueda crear los códigos.
\item
	Tras crear el directorio de caché al responder desde el servidor web para que las pruebas funcionen.
\end{itemize}

Para actualizar los permisos basta con que uses \texttt{chown} en el directorio más profundo:

\begin{lstlisting}[language=sh]
cd src/
chown "$USER":http cache/qr
\end{lstlisting}

Los permisos del directorio se definen a \texttt{775} por el servicio.
