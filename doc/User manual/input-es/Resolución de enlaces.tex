\chapter{Resolución de enlaces}\label{resolucion-de-enlaces}

\section{Fichero de configuración}\label{fichero-de-configuracion}

El fichero de configuración del servidor instalado en \S\ref{ficheros-de-configuracion} define los primeros pasos que el servidor debe realizar para responder a las peticiones de los usuarios.
Cada vez que un usuario intenta acceder al servicio, lo hace con una petición HTTPS a \url{https://sl.ugr.es/}.
Cuando el servidor recibe una petición, abre el bloque \texttt{location /} en el fichero de configuración y busca dentro una página que servir al usuario para responder a su petición:

\begin{lstlisting}
location / {
	try_files $uri /php/resolver.php?$args;
	index pages/index.php;
}
\end{lstlisting}

La directiva \texttt{try\_files} intenta encontrar una página correspondiente a la URI pedida.
Por eje,plo, si el usuario pide \texttt{https://sl.ugr.es/licence}, intentará buscar una página que pueda servirle para la URI \texttt{licence}.
Se garantiza que \texttt{\$uri} va a fallar, ya que todas las páginas se encuentran en el directorio \texttt{pages/}, por lo que cada vez que el usuario pida una página será redirigido a \texttt{/php/resolver.php}.
El sufijo \texttt{?\$args} indica que todos los argumentos pasados junto con la URI serán pasados al resolvedor para que puedan ser utilizados en las páginas a las que redirija.
Discutiremos el resolvedor con más profundidad en \S\ref{el-script-de-resolucion}.
La directiva \texttt{index} permite al servidor saber qué cargar cuando el usuario accede al servidor directamente, es decir, \url{sl.ugr.es}.

El bloque \texttt{location \mysim\char`\\.php\$} hace lo mismo que el anterior, pero en su lugar intenta cargar \texttt{\$fastcgi\_script\_name}.
Esta directiva se usa en lugar de \texttt{\$uri} porque \texttt{fastcgi} es el protocolo que carga el servicio PHP.

Por último, la directiva \texttt{error\_page} redirige todos los \texttt{404} a \texttt{/pages/404.php}.
En teoría, esta directiva no hace falta porque el resolvedor envía a los usuarios perdidos a la página 404, pero nadie va a quejarse por un poco de resiliencia y cargar una página 404 personalizada es mucho más elegante que servir la página por defecto.

\begin{lstlisting}
error_page 404 /pages/404.php;
\end{lstlisting}

\section{El \textit{script} de resolución}\label{el-script-de-resolucion}

Todas las peticiones se reenvían a este \textit{script}, que interpreta la URI y encuentra la mejor página para servir al usuario o lo redirige desde un enlace corto.
El \textit{script} fue escrito por razones de escalabilidad dados los requisitos de Neosluger:

\begin{itemize}
\item
	El usuario debe poder acceder a las páginas del sitio sin tener que saber su ruta absoluta.
	Por ejemplo, \texttt{sl.ugr.es/man} debe poder servir \texttt{pages/manual.php}.
\item
	El usuario debe poder usar un enlace corto para ser redirigido automáticamente y añadir \texttt{stats/} antes del identificador del enlace para inspeccionarlo y ver sus estadísticas de uso.
\item
	El usuario no debe poder acceder a las diferentes páginas desde su ubicación en el árbol de directorios del servidor.
	Esto mejora la seguridad del servicio.
\item
	La API antigua debe seguir funcionando aunque siempre lance un error.
\end{itemize}

Si las URIs fueran resueltas por el fichero de configuración, cada vez que se añadiese una nueva pagina al servicio habría que modificarlo y se abarrotaría muy rápido.
En su lugar, hacemos que el \textit{script} de resolución gestione esto por nosotros y compruebe si el usuario está accediendo a una página del servicio o quiere ser redirigido desde un enlace corto.

El resolvedor sigue los siguientes pasos para servirle al usuario la página que espera:

\begin{lstlisting}
Interpretar la URI recibida por el servidor

si encuentra una página coincidente en pages/*php
	mostrar la página
en otro caso, si la URI NO es de un enlace corto disponible en la base de datos:
	si se refiere a la API antigua:
		mostrar el mensaje de error de obsolescencia de la API
	en otro caso
		mostrar la página de 404
en otro caso, si el identificador se encuentra en la base de datos:
	Impedir al usuario almacenar la página en caché para registrar todos sus accesos
	Registrar el acceso al enlace
	Redirigir al usuario a su destino
\end{lstlisting}

El orden de los caminos es un poco extraño porque para comprobar si una URL no se encuentra en la base de datos hacemos una llamada a \texttt{\$url->is\_null()} y las condiciones de los \texttt{if} están escritas de forma que no contengan negaciones, de forma que sea más fácil razonarlas.

\section{Muestra de las páginas}\label{muestra-de-las-paginas}

Todas las páginas en \texttt{pages/*.php} \textbf{deben} contener una función \texttt{render} que actúe como la función principal de la página y llamarla al final del fichero.
De esta forma, cuando el resolvedor las incluya, la función \texttt{render} se llama automáticamente por la página.

Las páginas se componen usando Twig, un motor de plantillas que permite a Neosluger reutilizar código de las interfaces y lo hace mucho más fácil de mantener.
Las directivas de Twig se espefician dentro de bloques \texttt{\{\% \%\}} y \texttt{\{\{ \}\}}.
El primero tipo de bloques se utiliza para palabras reservadas como \texttt{[end]block}, \texttt{[end]if} y para llamar a las funciones miembro de los objetos como \texttt{url.is\_null()} (fíjate en el operador \texttt{.} en lugar de \texttt{->}).
El segundo tipo de bloques expande el valor entre las llaves.
Por eejemplo, si tenemos la variable \texttt{\$name = "Taxo"}, \texttt{<p>Mi nombre es \{\{ name \}\}.</p>} se expandirá a \texttt{<p>Mi nombre es Taxo.</p>}

Para mostrar una página hay que llamar a la función \texttt{Twig->render()} mediante un \texttt{echo}.
Esta función recibe la ruta del fichero HTML que debe mostrar y una lista asociativa de objetos cuyos nombres son las variables utilizadas por la plantilla.
